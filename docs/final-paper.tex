\documentclass[english]{article}
\usepackage[T1]{fontenc}
\usepackage[utf8]{inputenc}
\usepackage{geometry}
\geometry{verbose,tmargin=3.5cm,bmargin=4cm,lmargin=3.8cm,rmargin=3.8cm}

% plots
\usepackage{graphicx}
\graphicspath{ {../plots/} }

% newlines instead of indent for paragraphs
\usepackage[parfill]{parskip}
% tables
\usepackage{multirow}

% import natbib and sets bibliography and citation styles
\usepackage[numbers,sort]{natbib}
\bibliographystyle{apalike}

\makeatletter
\usepackage{hyperref}

\makeatother
\usepackage{babel}

% work in progress packages
\newcommand{\citationneeded}{\textsuperscript{\color{blue} [citation needed]}}
\usepackage{easy-todo}

\begin{document}

\listoftodos

\title{Keep your enemies closer and be loud about it}
\author{Martin Toman}
\date{06 May 2021}
\maketitle

\todo{add supervisor name}
\todo{add contact info and institution}


\begin{abstract}
% The abstract should be short and give the overall idea:
% what is the background, the research questions, what is contribution, and what are the main conclusions.
% It should be readable as a stand-alone text (preferably no references to the paper or outside literature).

\todo{write abstract}
\end{abstract}



\section{Introduction}

% intro - what is prisoner's dilemma
How can we encourage and sustain cooperation? Humans dominate their environments thanks to our ability to cooperate flexibly and at scale, as argued by \citet{harari-sapiens}.
To study the conditions necessary for cooperation to flourish we need a suitable model of an activity with temptations to defect and punishments for doing so.

In 1950, Albert Tucker named a particular two-player exchange game "The Prisoner's Dilemma" \citep{sep-prisoner-dilemma}.
This game elegantly captures the difficulty of the decision between cooperation and defection in a single choice.
Despite being so simple compared to the complexity of the problem it is representing, it was used to model many aspects of behaviour in systems of selfish individuals; and, according to \citet{Axelrod84}, for "discovery of the precise conditions that are necessary and sufficient for cooperation to emerge".

% iterated version + the dilemma
In the case of a one-off exchange, there being no opportunity for a follow-up punishment, the rational behaviour is defection. (This extends to all rounds for a fixed-length game, inductively \citep{Axelrod84}.)
The interesting behaviour arises if there is no end; or, at least, if there is no way for the participants of the game to know when the game ends or even if there is an end.
% reputation importance
An agents has to expect that even a single defection can be infinitely punished by never again being cooperated with \citep{GRIM}.
Such a risk may just not be worth it.

% global reputation system
The defectors can, naturally, only be punished if they can be identified and known to others. This is why services like Ebay or Airbnb have a rating system in place.
Presence of a reputation system has been shown to strongly boost cooperation, as shown by \citet{simple-reputation} and \citet{public-private-monitoring}.
These studies used groups of volunteers as game participants and explored the effects of various information being public - from only the latest move of the current opponent, to full histories of all moves taken by every participant.

% decentralizing reputation: keeping track internally?
These studies were limited by their use of humans as game participants and were thus limited to relatively small groups with few rounds; they also used external infrastructure for information passing: therefore eliminating noise, delays, and deliberately wrong information.
As shown by \citet{noise}, not all strategies that perform well in noise-less environments can do so under the presence of noise.

Using external infrastructure for passing information also meant that the transmission speed was uniform for all participants receiving all necessary information in time for their next round of the game.
These are non-trivial idealizations: relaxing them would yield a model closer to real-world systems and could change the results drastically.

In this paper we look at if and how well a local reputation system sustains cooperation and under what conditions does it yield optimal results.
We evaluate the approach under various gossip range and memory length; and comment on the effectiveness of local reputation in enforcing cooperation in spatial prisoner's dilemma.

\todo{outline the structure of the paper}


\section{Methodology}
We aim to measure the effectiveness of local reputation in enforcing cooperation. To do this we will build a computer simulation of a spatial multi-agent environment.

In this section we define the goals of this paper explicitly, present the design of the model and simulations used to verify the results, and present properties of the model which we will measure together with the evaluation criteria used to evaluate the effectiveness of our approach.

\subsection{Problem Statement}
We will use the prisoner's dilemma game to model the main interaction between the agents. This is a good choice for observing conditions necessary for cooperation to emerge in a population of rational and selfish agents.
\citationneeded

All agents will act independently; the only mutual interactions will be playing the game with a neighbor and exchanging gossip with nearby agents.
We will vary the range at which the gossip can be exchanges as well as the amount of information which can be included in as single gossip message.

\subsection{Simulation Design}
% Typically in general research articles, the second section contains a description of the research methodology, explaining what you, the researcher, is doing to answer the research question(s), and why you have chosen this method.
% This section includes references to necessary background information.

To explore the effects of local reputation, built up via openly gossiping with local peers, we will use a computer simulation of a multi-agent spatial environment. We will base it on the design of \citet{smaldino}.

\todo{explain the smaldino model here shortly}

A single round of the game is defined using a payoff matrix as shown in Table \ref{table:payoff}, with $T > R > P > S$ and $2R > T + S$ \citep{chammah1965}.

\begin{table}[h!]
  \centering
  \begin{tabular}{c c||c|c}
    & & \multicolumn{2}{c}{Opponent's move} \\
    & & Cooperate & Defect \\
    \hline\hline

    \multirow{4}{6em}{Player's move}
    & \multirow{2}{5em}{Cooperate}
      & Player:\ \ \ \ \ \ R & Player:\ \ \ \ \ \ S \\
    & & Opponent: R & Opponent: T \\
    \cline{2-4}
    & \multirow{2}{5em}{Defect}
      & Player:\ \ \ \ \ \ T & Player:\ \ \ \ \ \ P \\
    & & Opponent: S & Opponent: P \\
  \end{tabular}

  \caption{Payoff matrix}
  \label{table:payoff}
\end{table}

Participants are agents living in a 2D grid, accumulating energy via repeatably engaging in rounds of PD as defined above. Agents are moving at the same speed, and their order in which they take their turn is randomly determined at every clock tick. They pay a fixed cost to survive to a next round, which is subtracted from all peers at the end of each turn (agents who deplete their energy die and are removed from the simulation).
Another use of the energy is to constrain reproduction and allow only the most successful agents to reproduce, producing an offspring with an identical strategy.
\todo{rewrite and connect the preceding paragraph}

To better suit our needs we will reduce the spatial grid size from 100x100, as in the original design, to a more manageable 20x20, to compensate for more complex agent behaviour and to keep the simulation times low. We will also decrease the starting number of agents from 1600 to 64: keeping the same ratio of 16\% of the total grid size.

This reduction of the environment size (by a factor of 25!) has the effect of greatly increasing the chance of the total extinction of all agents: caused by the random behaviour of agents exploiting each other until all cooperators are dead and the population of pure defectors cannot sustain itself.
We disregard runs that end in extinction and increase the number simulations run to compensate for this.

We will expand the model by giving the agents a (limited size) memory to keep track of past defectors and later to allow them to actively and freely share this knowledge by gossiping with other agents in a given range.
\todo{add references to related work on memory}

\subsection{Simulation Evaluation}

To determine the effectives of gossip in enforcing cooperation, we will observe the rate of convergence to a population of cooperators, stopping the simulation once stable equilibrium is achieved. We will also observe the maximal population size of defectors which can sustain itself alongside the cooperators.
We will also record characteristic patterns formed by the populations as influenced by different parameters.

\todo{show how we evaluation the effectiveness}



\section{Results}
% As discussed earlier, in many sciences the methodology is explained in section 2 and this section only discusses the results.
% However, in computer science, most often the details of the evaluation setup are described here first (simulation environment, etc.).
% Very important here is that any skilled reader would be able to reproduce this setup and then obtain the same results.

% Then, results are reported in an accessible manner through figures (preferably with captions that allow them to be understood without going through the whole text), observations are made that clearly follow from the presented results.
% Conclusions are drawn that follow logically from the previous material.
% Sometimes the conclusions are in fact hypotheses, which in turn may give rise to new experiments to be validated.

\begin{figure}[h!]
  \centering
  \includegraphics{frequency_time_with_memory.pdf}
  \caption{Evolution of agent population over time (memory size = 1)}
  \label{table:population_evolution}
\end{figure}

\begin{figure}[h!]
  \centering
  \includegraphics{cooperator_frequency_memory_50steps.pdf}
  \caption{Agent type frequency after $50$ steps}
  \label{table:cooperator_frequency_converged_50}
\end{figure}
\begin{figure}[h!]
  \centering
  \includegraphics{cooperator_frequency_memory_100steps.pdf}
  \caption{Agent type frequency after $100$ steps}
  \label{table:cooperator_frequency_converged_100}
\end{figure}

% You may want to give this section another name.
\todo{name this section}



\section{Discussion}
% Results can be compared to known results and placed in a broader context.
% Provide a reflection on what has been concluded and how this was done.
% Then give a further possible explanation of results.

% You may give this section another name, or merge it with the one before or the one hereafter.
\todo{write discussion: compare with the human reputation experiment}



\section{Responsible Research}
\todo{write this section: small open source model, jupyter notebook (self contained), nix}

% Reflect on the ethical aspects of your research and discuss the reproducibility of your methods.



\section{Conclusions and Future Work}
% Summarize the research question(s) and the answers to the research question(s).
% Make statements.
% Highlight interesting elements.

% Discuss open issues, possible improvements, and new questions that arise from this work; formulate recommendations for further research.

% ideally, this section can stand on its own: it should be readable without having read the earlier sections.
\subsection{Our conclusions}
\todo{conclude}

\subsection{Future work}
\todo{outline future work: noise, more complex agents, p2p?, make the model public}



\pagebreak
\bibliography{references}

\end{document}
